\documentclass[11pt]{article}

\usepackage[a4paper, margin=1in]{geometry}
\usepackage[british]{babel}
\usepackage[backend=biber, style=authoryear-icomp]{biblatex}

\addbibresource{bibliography.bib}

\title{Lambda Calculus and its Impact on Computer Science}
\author{Joris Klaasse Bos\\ Zaanlands Lyceum}

\begin{document}

\maketitle
\newpage
\tableofcontents
\newpage

% I still need to decide wether I need a preface and afterword

\section{Preface}

% TODO: Something about PWS

The reason I chose this subject is that it combines two things I enjoy:
abstract maths and computer science. I have been programming for about five to
six years now. I mainly enjoy low-level programming, so, naturally, C is my
most used language and I am most familiar with a simple procedural paradigm.
Such a paradigm is, however, not always very easy to use when working with very
large and complex systems. I, as many, started out with Object-Oriented
Programming, but I did not like that very much. Therefore, I have been
exploring alternative paradigms, including Data-Oriented Programming and
Functional Programming. I am quite familiar with Data-Oriented Programming and
the Rust programming language by now, but Functional Programming isn't
something I've ever really got into yet. I did find out about lambda calculus
and combinatory logic, which intrigued me, but I haven’t got into it beyond a
basic level of understanding. That is why I decided to research it for this
paper. 

\section{Introduction}

With the decline of OOP, many other paradigms are gaining in popularity. One
increasingly popular paradigm is functional programming. Functional programming
is fundamentally based on lambda calculus and it has been seeping into other
paradigms and into mainstream languages. Most of the popular languages now
implement lambda functions and have ways to write in a more declarative style
of programming. In this paper I will look at all the ways lambda calculus is
being used in computer science and how it may be used in the future.

\section{Introduction to lambda calculus}

Lambda calculus is, as its name suggests, a calculus. A calculus is a system of
manipulating symbols, which by themselves don't have any semantic meaning, in a
way that is somehow meaningful. We all know algebra. Algebra itself doesn't
have an innate meaning, but we can use it to represent and solve real world
problems. Algebra, however, is limited. Not every problem can be represented in
algebra. There are many branches of mathematics that use different systems.
One example would be formal logic, which is used for logical operations on
booleans. Another such example is lambda calculus.

\subsection{A short history}

% TODO: Something about church, turing, church-turing hypothesis and formal
% equivalency

\subsection{The syntax}

Lambda calculus is all about first-class higher-order pure
(anonymous\footnote{The core lambda calculus has no way of naming functions.})
unary functions. Such a function takes a single input, and returns a single
expression that is only dependent on the input, so it doesn't have any outside
state. Such a function can take and return any expression, which in lambda
calculus is always a function. A simple function definition in lambda calculus
looks as follows: \[\lambda a.a\]

The lambda signifies a function. Everything following it will be part of that
function's definition. The \(a\) before the \(.\) is the name of the argument.
There is only one, because, as I said before, all functions in lambda calculus
are unary. Everything following the \(.\) is part of the function body, which
is the return expression. The funcion above is the identity function in lambda
calculus; it just returns the input. This is the equivalent of multiplying by
one, or defining a function like \(f(x)=x\), or multiplying a vector with the
identity matrix; it does nothing.

But how do we use this function? Well, just like defining a function, it is
quite simple. If you want to apply this function to a symbol, you just put it
in front of the symbol. Something like this:
\[(\lambda a.a)x\]
Which evaluates to \(x\), because you remove the \(x\) and then replace all the
\(a\)'s in the function body with \(x\) and then remove the function signifier
and argument list. It is a simple process of substitution.

In this case you need parentheses around the function, otherwise \(x\) would be
considered part of the function's body, which it isn't. It's also important to
note that lambda calculus is left-associative, that is, it evaluates an
expression from left to right. This means that the function on the far left of
an expression gets invoked first.

I have now basically explained the entire lambda calculus, it is really that
simple. I have explained abstraction (functions), application (applying
functions), and grouping (parentheses), which is basically all we need. You
can also give names to expressions. We could name our identity function \(I\)
as follows: \[I:=\lambda a.a\] But this isn't really part of the core lambda
calculus anymore, just some syntactic sugar. This way, instead of constantly
having to write \(\lambda a.a\), we can just write \(I\). So instead of
writing: \[(\lambda a.a)x=x\] We could use our previous definition of \(I\) and
write: \[Ix=x\] We have now covered identifiers too.

But if this is all there is, how can this possibly be Turing complete? How do
we do boolean logic, or algebra? How can we do things with only unary
functions? What are \(a\) and \(x\) supposed to represent? If there is no
concept of value, how do we even use this meaningfully?

Well, the key is this: a function can return any expression, so even other
functions, not just a single symbol. We can start composing these simple
functions into more complex functions. Let's say that we wanted to have a
function that takes two arguments, and then applies the first argument to the
second one. You are probably asking yourself a few questions. For example, what
does it mean for one argument to be applied to another? Well, as I said, these
arguments are expressions and can thus be functions themselves. But the biggest
question you are probably asking yourself is: how can you have a function that
takes two arguments?

Well, we actually can't, but what we can do is to have a function that takes
one argument and returns another function that takes one argument. We can
define the function as follows:
\[\lambda a.\lambda b.ab\]
We currently have a function definition inside the body of another function. If
we now apply this function to a symbol like \(x\), we get this:
\[(\lambda a.\lambda b.ab)x=\lambda b.xb\]
We get a new function that takes an argument and applies \(x\) to it. If we now
apply this function to a symbol like \(y\), we get this:
\[(\lambda b.xb)y=xy\]
Alternatively, we could write it all on one line:
\[(\lambda a.\lambda b.ab)xy=(\lambda b.xb)y=xy\]

\(xy\) in this case is what we would call the \emph{\(\beta\)-normal form} of
the preceding expressions. That just means that it is in the simpelest form and
isn't able to be evaluated any further. Reducing an lambda expression to the
\(\beta\)-normal form is called \emph{\(\beta\)-reduction}.

You can start to see how we can compose unary functions to create more complex
functions\footnote{That's what makes them higher-order (and first-class).}. In
this example we used two nested unary functions to get the same result you
would with a binary function. Such a nested function is often called a
\emph{curry'd function}.

You might think to yourself that having this many nested functions can be quite
convoluted and not very readable, and you're quite right. That's why people
often use a shorthand notation. They would basically write it as if it is a
single binary function (as with any n-ary function). They would write the
example function above as:
\[\lambda ab.ab\]
Do keep in mind, that even though this looks and, for the most part, acts as if
it is a single binary function, it really isn't. It still is a curry'd function
that feeds in the arguments one-by-one, but this way the expression becomes
more readable and easier to think about conceptually.

Congratulations, you now know the very basics of lambda calculus. You may still
not see how this is Turing complete or how this can be useful and meaningful.
You might also already see some of the intrigues of lambda calculus; how simple
it is, how it doesn't have a concept of value or data, how everything is an
expression, how it is stateless, etc. But we'll get to all of that eventually.
If you get this, everything else will follow naturally (mostly).

\subsection{Combinatory logic}

Combinatory logic is a notation to eliminate the need for quantified variables
in mathematical logic \parencite{wiki:Combinatory_logic}. That basically means
a form of logic without values, just like in lambda calculus, but just pure
logical expressions, using so called \emph{combinators}. The idea of
combinators first came from \textcite{schonfinkel1924}, and was later
rediscovered by \textcite{curry1930}. Combinators are just symbols, in this
case letters, that perform operations on symbols that succeed it. We've
actually looked at one of these combinators already.

\subsubsection{Identity}

The first combinator we'll cover is \(I\). It does exactly the same thing as
our \(I\) function we defined in lambda calculus in the previous subsection
(\(I:=\lambda a.a\)). In fact, all combinators can be defined in lambda
calculus. This combinator may seem quite useless, but it is actually quite
useful when composing combinators, which we'll come to soon.

\subsubsection{The omega combinator}

The next combinator we'll cover is \(M\). All it does is repeat its one
argument twice. It can be defined in lambda calculus as follows:
\[M:=\lambda f.ff\]

We could, for example, look at what happens when you apply \(M\) to \(I\). We
get:
\[M I = I I = I\]
Or, written out in lambda calculus:
\[(\lambda f.ff)\lambda a.a=(\lambda a.a)\lambda a.a=\lambda a.a\]

What happens if you apply \(M\) to \(M\)? You get:
\[M M = M M = M M = ...\]
and so on to infinity. Or in lambda calculus:
\[(\lambda f.ff)\lambda f.ff=(\lambda f.ff)\lambda f.ff=(\lambda f.ff)\lambda f.ff=...\]

This expression cannot be evaluated. We say that it doesn't have a
\(\beta\)-normal form. In lambda calculus and combinatory logic not every
expression is reducable. As \textcites{turing1936}{turing1937}
proved\footnote{This is not what he proved specifically.}: there is no single
algorithm to decide wether a lambda expression has a \(\beta\)-normal form.
This is called the \emph{halting problem}\footnote{To be precise: In
computability theory, the halting problem is the problem of determining, from a
description of an arbitrary computer program and an input, whether the program
will finish running, or continue to run forever
\parencite{wiki:Halting_problem}.}.

\(M M\) is sometimes called the \(\Omega\) combinator. Omega, because it is the
end of the Greek alphabet. The \(M\) combinator is sometimes called the
\(\omega\) combinator because of this. These combinators often have many
different names. Sometimes because scientists discovered them seperatly,
unaware of eachother, sometimes because scientists like to give them pet names.

\subsubsection{The constant combinator}

The next combinator we'll cover is \(K\). It is a combinator that takes two
arguments and returns the first. We can easily define it in lambda calculus as
follows:
\[K:=\lambda ab.a\]

Remember that we defined this as a curry'd function. This means we can give it
just one argument and get a new function out of it. Let's say we apply \(K\) to
\(5\):
\[K5=(\lambda ab.a)5=\lambda b.5\]
Our new function, \(K5\), is a function that takes an argument and returns
\(5\). This means that whatever we apply this function to, we always get \(5\).
\(K\) gets its name from the German word \emph{Konstant}, meaning constant. You
can probably see why.

Just like with the previous combinators, it'll prove very useful, much more so
than you'd expect.

\subsubsection{The kite}

Here is where things get a little spicier. Our next combinator is \(KI\). It
takes two arguments and returns the latter. We can define it in lambda calculus
as follows:
\[KI:=\lambda ab.b\]

You may already be thinking about its name. Why does it have two letters? And
why are they two letters we've talked about already? Well, the answer is very
simple. If you apply \(K\) to \(I\), you get \(KI\). Don't believe me? Let's
try!

If we use our definition of \(KI\) and apply it to \(xy\) we get:
\[KIxy=(\lambda ab.b)xy=y\]
But if we use the K and I combinators seperately, we get the following:
\[KIxy=(\lambda ab.a)Ixy=(\lambda b.I)xy=Iy=y\]

If you think about it, it is very logical. If \(K\) takes two arguments and
returns the first, then, in this case, it uses up both \(I\) and \(x\) and
returns \(I\), which will just return the next argument, in this case \(y\).
\(KI\) will always return the second symbol after the I, because---again---the
first gets used up by \(K\).

We can also just see what function we get when we apply \(K\) to \(I\):
\[KI=(\lambda ab.a)I=\lambda b.I=\lambda b.\lambda a.a=\lambda ba.a\] We get
our definition of \(KI\) (except the names of the arguments are switched).

We're starting to define combinators as combinations of other combinators.
Every combinator, in fact, can be defined as a combination of other
combinators. That's why they are called combinators.

\subsubsection{The starling}

\subsubsection{The cardinal}

\subsubsection{The bluebird}

\subsubsection{\(S\) and \(K\)}

\subsubsection{To Mock a Mockingbird}

\subsubsection{Equality}

\section{Using lambda calculus for computation}

\subsection{Binary logic}

\subsection{Church numerals}

\section{Functional programming (lambda calculus applied)}

\subsection{Laziness}

% Note to self: write about how lambda functions can create new lambda
% functions, as we've seen in the combinatory logic subsection

\subsection{Haskell}

\section{Functional programming in other paradigms}

\section{Possibilities for the future}

\section{Afterword}

\newpage
\printbibliography[heading=bibintoc, title={References}]
\end{document}
