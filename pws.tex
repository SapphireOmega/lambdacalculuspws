\documentclass[11pt]{article}

\usepackage[a4paper, margin=1in]{geometry}
\usepackage[british]{babel}
\usepackage[backend=biber, style=numeric-comp]{biblatex}

\addbibresource{bibliography.bib}

\title{Lambda Calculus and its Impact on Computer Science}
\author{Joris Klaasse Bos\\ Zaanlands Lyceum}

\begin{document}

\maketitle
\newpage

\section*{Preface}
\addcontentsline{toc}{section}{Preface}

% TODO: Something about PWS

The reason I chose to write about this subject, is that it combines two things
I enjoy: abstract maths and computer science. I have been programming for about
five to six years now. I mainly enjoy low-level programming, so---naturally---C
is my most used language and I am most familiar with a simple procedural
paradigm. Such a paradigm is, however, not always very easy to use when working
with very large and complex systems. I, as many, started out with
object-oriented programming, but I did not like that very much. Therefore, I
have been exploring alternative paradigms, including data-oriented programming
and functional programming. I am quite familiar with data-oriented programming
and the Rust programming language by now, but functional programming isn't
something I've ever really got into yet. I did find out about lambda calculus
and combinatory logic, which intrigued me, but I haven’t got into it beyond a
basic level of understanding. That is why I decided to research it for this
work. 

\newpage

\tableofcontents
\newpage

\section{Introduction}

With the decline of OOP, many other paradigms are gaining in popularity. One
increasingly popular paradigm is functional programming. Functional programming
is fundamentally based on lambda calculus and it has been seeping into other
paradigms and into mainstream languages. Most of the popular languages now
implement lambda functions and have ways to write in a more declarative style
of programming. In this work I will look at all the ways lambda calculus has
influenced computer science and how it may do so in the future.

\section{Introduction to lambda calculus}

Lambda calculus is, as its name suggests, a calculus. A calculus is a system of
manipulating symbols, which by themselves don't have any semantic meaning, in a
way that is somehow meaningful. We all know algebra. Algebra itself doesn't
have an innate meaning, but we can use it to represent and solve real world
problems. Algebra, however, is limited. Not every problem can be represented in
algebra. There are many branches of mathematics that use different systems.
One example would be formal logic, which is used for logical operations on
booleans. Another such example is lambda calculus.

\subsection{A short history\label{history}}

People always trusted mathematics to be true and relied on it heavily. If
something was proven true with mathematical logic, then that must be true.
However, starting from the late 19th century, people ran into paradoxes. People
made a distinction between reasoning that is rigorous and reasoning that
isn't---reasoning that is logical, and reasoning that is psychological. The
fact that mathematical logic, which people looked at for rigorousness, is
infested with paradoxes and self-reference was very troubling for people at the
time.

The concept of mathematics and mathematical logic wasn't well defined, so
people started to think about the formalisation of mathematical logic to try to
solve these issues. People wanted a system that would encapsulate all of
mathematical logic. Preferably this system would be simple, clean and
intuïtive.

Throughout the late 19th and early 20th centuries, people started formally
defining and redefining different aspects of mathematics. \textcite{frege1879}
wrote about propositional calculus and functions as graphs, and in doing so
reëvaluated the concept of functions and was already using concepts like
Currying functions (more on this in section \ref{syntax}) without really giving
thought to it. \textcite{peano1889} invented the Peano axioms and Peano
arithmetic as a way of defining natural numbers. He was not the first to
attempt defining natural numbers, but he was the most successful.
\textcite{schonfinkel1924} invented combinatory logic, which was later
rediscovered and improved on by \textcite{curry1930}, as a way to remove the
need for quantified variables in logic.

One major attempt to define all of mathematics was done by
\textcite{russell1997}. They wrote a book that would become well know in all of
mathematics and logic. This book is called \emph{Principia Mathematica}. They
did, however, run into a few problems, which arose from self-reference. To
solve these problems that this paradoxical self-reference brought with it, they
invented an elaborate system, the theory of types, to circumvent/eliminate it.
It was a very carefully crafted bastion against self-reference ever coming up
in their system, which was not very simple, clean or intuïtive.

People praised \emph{PM} as they thought they had finally done it; they had
formalised all of mathematical logic, they had realised the dream of grounding
all of mathematics in logic. But in Vienna, Gödel was sceptical of this book.
He started seeing some cracks, he felt that there was something wrong aboout
this attempt. Gödel felt that self-reference was a fundamental part of
mathematical logic. Then he went out and actually proved that there is no
consistent system of axioms whose theorems can be listed by an effective
procedure that is capable of proving all truths about arithmetic of natural
numbers\footnote{This is a definition of the first incompleteness theorem I got
from \textcite{wiki:Incompleteness_theorems}} \parencite{godel1931}, meaning
that such a system is either inconsistent or incomplete\footnote{Incompleteness
means that there are things that are true, but are not provable.}, greatly
disturbing many mathematicians and upending mathematics as they knew it.

During this time, in this environment of the formalisation of mathematical
logic, \textcite{church1932} invented the lambda calculus. Lambda calculus is a
very simple and minimalistic system of substitution. A little while later,
\textcite{turing1936, turing1937correction} invented Turing machines. Turing
machines are conceptual mathematical machines that function based on
state---they were state machines. These could perform all kinds of mathematical
and logical computations. He was not the first to invent computers, but he was
the first to work them out as well as he did (and, as you probably know, he
built one which he cracked the German enigma code with).

There was this problem that has a few different names. It is often known as the
\emph{halting problem} or the \emph{Entscheidungsproblem}, which is German for
\emph{decision problem}. The halting problem and decision problem aren't
exactly synonyms, but they come down to the same thing. Basically, it asks
wether it is possible to know via an algorithm wether a computation will
complete execution or result in an infinite loop. In 1936,
\textcite{turing1936, turing1937correction} spent a long time proving, using
his Turing machines, that this isn't possible, but it didn't get published
until early 1937. Also in 1936, \textcite{church1936} proved the same thing
using lambda calculus and happended to publish it before Turing did. When
Turing finally got around to publishing his proof, he found out that he was
beaten to it by Church. He wasn't to pleased. What is interesting, though, is
that lambda calculus and Turing machines take two completely different
approaches. Turing machines funtion entirely on state, while lambda calculus is
completely stateless (we'll look at this later). Turing thought this was
interesting too, so he researched lambda calculus and how it relates to his
Turing machines, and proved that they are formally equivalent
\parencite{turing1937computability}. 

Why do I tell you all this? Well, your main takeaway should be that even though
lambda calculus is a very simple system, which at first glance might not seem
to be very semantic or seem to have any real world implications, it actually is
Turing complete. Lambda calculus and Turing machines take wildly different
approaches: one state based, the other stateless. Another difference is that
Turing machines can be physically built. We can, however, use lambda calculus
on these Turing machines \emph{and} simulate Turing machines with lambda
calculus, which is a fundamental part to the thesis of this work. We will look
at lambda calculus and how the work of all the previously mentioned
mathematicians, and many more, can be applied in lambda calculus to get a
useful system.

\subsection{The syntax}\label{syntax}

Lambda calculus is all about first-class higher-order pure
(anonymous)\footnote{The core lambda calculus has no way of naming functions.}
unary functions. Such a function takes a single input, and returns a single
expression that is only dependent on the input, so it doesn't have any outside
state. Such a function can take and return any expression, which in lambda
calculus is always a function. A simple function definition in lambda calculus
looks as follows: \[\lambda a.a\]

The lambda signifies a function. Everything following it will be part of that
function's definition. The \(a\) before the \(.\) is the name of the argument.
There is only one, because, as I said before, all functions in lambda calculus
are unary. Everything following the \(.\) is part of the function body, which
is the return expression. The funcion above is the identity function in lambda
calculus; it just returns the input. This is the equivalent of multiplying by
one, or defining a function like \(f(x)=x\), or multiplying a vector with the
identity matrix; it does nothing.

But how do we use this function? Well, just like defining a function, it is
quite simple. If you want to apply this function to a symbol, you just put it
in front of the symbol. Something like this:
\[(\lambda a.a)x\]
Which evaluates to \(x\), because you remove the \(x\) and then replace all the
\(a\)'s in the function body with \(x\) and then remove the function signifier
and argument list. It all comes down to a simple process of substitution.

In this case you need parentheses around the function, otherwise \(x\) would be
considered part of the function's body, which it isn't. It's also important to
note that lambda calculus is left-associative, that is, it evaluates an
expression from left to right. This means that the function on the far left of
an expression gets invoked first.

I have now basically explained the entire lambda calculus, it is really that
simple. I have explained abstraction (functions), application (applying
functions), and grouping (parentheses), which is basically all we need. You
can also give names to expressions. We could name our identity function \(I\)
as follows: \[I:=\lambda a.a\] But this isn't really part of the core lambda
calculus anymore, just some syntactic sugar. This way, instead of constantly
having to write \(\lambda a.a\), we can just write \(I\). So instead of
writing: \[(\lambda a.a)x=x\] We could use our previous definition of \(I\) and
write: \[Ix=x\] We have now covered identifiers too.

But if this is all there is, how can this possibly be Turing complete? How do
we do boolean logic, or algebra? How can we do things with only unary
functions? What are \(a\) and \(x\) supposed to represent? If there is no
concept of value, how do we even use this meaningfully? Well, the key is this:
a function can return any expression (remember?), which is always a
function\footnote{Everything is.}, not just a single symbol. We can start
composing these simple functions into more complex functions. Let's say that we
wanted to have a function that takes two arguments, and then applies the first
argument to the second one. You are probably asking yourself a few questions.
For example, what does it mean for one argument to be applied to another? Well,
as I said, everything is a function. But the biggest question you are probably
asking yourself is: how can you have a function that takes two arguments?

We actually can't, but what we can do is to have a function that takes one
argument and returns another function that takes one argument. We can define
that function as follows:
\[\lambda a.\lambda b.ab\]
We currently have a function definition inside the body of another function. If
we now apply this function to a symbol like \(x\), we get this:
\[(\lambda a.\lambda b.ab)x=\lambda b.xb\]
We get a new function that takes an argument and applies \(x\) to it. If we now
apply this function to a symbol like \(y\), we get this:
\[(\lambda b.xb)y=xy\]
Alternatively, we could write it all on one line:
\[(\lambda a.\lambda b.ab)xy=(\lambda b.xb)y=xy\]
\(xy\) in this case is what we would call the \emph{\(\beta\)-normal form} of
the preceding expressions. That just means that it is in the simpelest form and
isn't able to be evaluated any further. Reducing a lambda expression to the
\(\beta\)-normal form is called \emph{\(\beta\)-reduction}.

You can start to see how we can combine unary functions to create more complex
functions\footnote{That's what makes them higher-order (and first-class).}. In
this example we used two nested unary functions to get the same result you
would with a binary function. Such a nested function is often called a
\emph{Curry'd function}.
You might think to yourself that having this many nested functions can be quite
convoluted and not very readable, and you're quite right. That's why people
often use a shorthand notation. They would basically write it as if it is a
single binary function (as with any n-ary function). They would write the
example function above as:
\[\lambda ab.ab\]
Do keep in mind, that even though this looks and, for the most part, acts as if
it is a single binary function, it really isn't. It still is a Curry'd function
that feeds in the arguments one by one, but this way the expression becomes
more readable and easier to think about conceptually. We will use this notation
from now on.

Congratulations, you now know the very basics of lambda calculus. You may still
not see how this is Turing complete or how this can be useful and meaningful.
You might also already see some of the intrigues of lambda calculus; how simple
it is, how it doesn't have a concept of value or data, how everything is an
expression, how it is stateless, etc. But we'll get to all of that eventually.
If you get this, everything else will follow naturally (mostly).

\subsection{Combinatory logic}\label{combinatorylogic}

Combinatory logic is a notation to eliminate the need for quantified variables
in mathematical logic \parencite{wiki:Combinatory_logic}. That basically means
a form of logic without values, just like with lambda calculus, but just pure
logical expressions, using so called \emph{combinators}. The idea of
combinators first came from \textcite{schonfinkel1924}, and was later
rediscovered by \textcite{curry1930}. Combinators are just symbols, in this
case letters, that perform operations on symbols that succeed it. We've
actually looked at one of these combinators already.

We will be using Curry's names of the combinators, since his names are most
widely used.

\subsubsection{Identity}

The first combinator we'll cover is \(I\). It does exactly the same thing as
our \(I\) function we defined in lambda calculus in the previous subsection
(\(I:=\lambda a.a\)). In fact, all combinators can be defined in lambda
calculus. Lambda calculus is really just 90\% combinatory logic, but without
identifiers. This combinator may seem quite useless, but it is actually quite
useful when composing combinators, which we'll come to soon.

\subsubsection{The omega combinator}

The next combinator we'll cover is \(M\). All it does is repeat its one
argument twice. It can be defined in lambda calculus as follows:
\[M:=\lambda f.ff\]

We could, for example, look at what happens when you apply \(M\) to \(I\). We
get:
\[M I = I I = I\]
Or, written out in lambda calculus:
\[(\lambda f.ff)\lambda a.a=(\lambda a.a)\lambda a.a=\lambda a.a\]

What happens if you apply \(M\) to \(M\)? You get:
\[M M = M M = M M = ...\]
ad infinitum. Or in lambda calculus:
\[(\lambda f.ff)\lambda f.ff=(\lambda f.ff)\lambda f.ff=(\lambda f.ff)\lambda f.ff=...\]

This expression cannot be evaluated. We say that it doesn't have a
\(\beta\)-normal form. In lambda calculus and combinatory logic not every
expression is reducable. As we've seen in the second to last paragraph of
section \ref{history}: there is no single algorithm to decide wether a lambda
expression has a \(\beta\)-normal form\footnote{This is not exactly what is
written, but it means the same thing.}.

\(M M\) is sometimes called the \(\Omega\) combinator. Omega, because it is the
end of the Greek alphabet. The \(M\) combinator is sometimes called the
\(\omega\) combinator because of this. Combinators often have many different
names. Sometimes because scientists discovered them seperatly, unaware of
eachother, sometimes because they preferred a different name, sometimes because
scientists like to give them pet names \footnote{I have a theory they are just
trying to throw us off}.

\subsubsection{The constant combinator}

The next combinator we'll cover is \(K\). It is a combinator that takes two
arguments and returns the first. We can easily define it in lambda calculus as
follows:
\[K:=\lambda ab.a\]

Remember that we defined this as a Curry'd function. This means we can give it
just one argument and get a new function out of it. Let's say we apply \(K\) to
\(5\):
\[K5=(\lambda ab.a)5=\lambda b.5\]
Our new function, \(K5\), is a function that takes an argument and returns
\(5\). This means that whatever we apply this function to, we always get \(5\).
\(K\) gets its name from the German word \emph{Konstant}, meaning constant. You
can probably see why.

Just like with the previous combinators, it'll prove very useful, much more so
than you'd expect.

\subsubsection{The kite}

Here is where things get a little spicier. Our next combinator is \(KI\). It
takes two arguments and returns the latter. We can define it in lambda calculus
as follows:
\[KI:=\lambda ab.b\]

You may already be thinking about its name. Why does it have two letters? And
why are they two letters we've talked about already? Well, the answer is very
simple. If you apply \(K\) to \(I\), you get \(KI\). Don't believe me? Let's
try!

If we use our definition of \(KI\) and apply it to \(xy\) we get:
\[KIxy=(\lambda ab.b)xy=y\]
But if we use the K and I combinators seperately, we get the following:
\[KIxy=(\lambda ab.a)Ixy=(\lambda b.I)xy=Iy=y\]

If you think about it, it is very logical. If \(K\) takes two arguments and
returns the first, then, in this case, it uses up both \(I\) and \(x\) and
returns \(I\), which will just return the next argument, in this case \(y\).
\(KI\) will always return the second symbol after the I, because---again---the
first gets used up by \(K\).

We can also just see what function we get when we apply \(K\) to \(I\):
\[KI=(\lambda ab.a)I=\lambda b.I=\lambda b.\lambda a.a=\lambda ba.a\] We get
our definition of \(KI\) (except the names of the arguments are switched).

We're starting to define combinators as combinations of other combinators.
Every combinator, in fact, can be defined as a combination of other
combinators. That's why they are called combinators.

\subsubsection{The flip combinator}\label{flipcombinator}

The next combinator is is the \(C\) combinator. The \(C\) combinator is
definable in lambda calculus as:
\[C:=\lambda fab.fba\]
What it basically does is switch the arguments to the next combinator around.

If we apply \(C\) to \(K\) and two random symbols, we get the same result we
would get if we had applied \(KI\) to those same symbols:
\[CKxy=Kyx=y\]
\[KIxy=y\]
\[CK=KI\]
Let's see what happens when we apply C to K in lambda calculus (I have changed
the names of \(K\)'s arguments as to avoid confusion with those of \(C\)):
\[(\lambda fab.fba)\lambda xy.x=\lambda ab.(\lambda xy.x)ba\]
We don't get our exact definition of \(KI\). But we can see that for every
input, \(CK\) and \(KI\) \emph{always} produce the same output. We say that
these functions are \emph{extentionally equal}---they have been defined
seperately and we cannot rewrite one to the other, but we know that they
produce the same results, so they must be equal.

You can do the same thing to find out that \(CKI=K\). It really does make
sense. \(K\) and \(KI\) both "select" one of two arguments. One selects the
first, the other selects the latter. Flipping their arguments make them select
the opposite of what they normally would, so they select the argument that the
other combinator usually would.

\subsubsection{The bluebird}

Our next combinator, \(B\), is defined as follows:
\[B:=\lambda fga.f(ga)\]
It applies a to b before applying f to the result. This combinator is used for
function composition. We say that the resulting function of the application of
B to two functions composes those two functions.

\subsubsection{The thrush}

Our next combinator is \(Th\). It is defined as follows:
\[Th:=\lambda af.fa\]
It swaps around two functions.
% todo stuff

\subsubsection{The starling}

Our last combinator is \(S\). It can be defined as follows:
\[S:=\lambda fga.fa(ga)\]
It applies f to a and its result to the result of the application of g to a.

\subsubsection{\(S\) and \(K\)}

% todo ref "every combinator can be ...."
% todo ref proof S and K completeness

As I've said, every combinator can be defined as a combination of other
combinators. The question arises: how many combinators do we need to define
every other combinator? It turns out you need just two. You can define every
other combinator using just \(S\) and \(K\).

% todo stuff

\subsubsection{To Mock a Mockingbird}

At the start of this section about combinatory logic, I said that
\textcite{schonfinkel1924} invented combinatory logic as a way of removing the
need for quantifiable variables. He started with propositional logic and
stripped it down until there was a very pure and simple form of logic left. But
how can we use this form of logic in the real world, if he even removes things
like propositions? You already know that it is Turing complete, so it must be
able to do any computation, but I haven't explain how yet. But we can use
combinatory logic in the real world already.

You may have noticed some of the previous subsections have birdnames as titles.
This is because they are the names given to the combinators, discussed in the
respective subsections, by an author named \textcite{smullyan2000}. He is a
mathematician who likes to write puzzle books. His book \emph{To Mock a
Mockingbird} \parencite{smullyan2000} is practically a large metaphor for
combinatory logic. There are some unrelated puzzles in the beginning of the
book, but the rest is about a big forest with birds. The birds represent the
combinators. The beginletters of the birdnames are the names of the
combinators. The way the birds interact reflects the way the combinators
interact. The reason he chose birds for his metaphor is because Curry was an
avid bird watcher. If you feel like you still don't understand the notation of
combinatory logic completely, I would recommend you give this book a read,
because it explains it very simply and clearly

I think it would be fun if we had a look at one of the puzzles to see if we can
solve it using our newfound knowledge of combinatory logic. I think the first
puzzle is sufficiently interesting. So far, \textcite{smullyan2000} has only
introduced the mockingbird and the idea of function composition, but not yet
the bluebird. I have taken the puzzle directly from the book:

\setlength{\leftskip}{1cm}
\setlength{\rightskip}{1cm}
\begin{center}
\rule{15cm}{0.5pt}
\end{center}

It could happen that if you call out \(B\) to \(A\), \(A\) might call the same
bird \(B\) back to you. If this happens, it indicates that \(A\) is fond of the
bird \(B\). In symbols, \(A\) is fond of \(B\) means that: \(AB = B\)

We are now given that the forest satisfies the following two conditions. 

\(C_{1}\) (the composition condition): For any two birds \(A\) and \(B\)
(whether the same or different) there is a bird \(C\) such that for any bird
\(x\), \(Cx = A(Bx)\). In other words, for any birds \(A\) and \(B\) there is a
bird \(C\) that composes \(A\) with \(B\). 

\(C_{2}\) (the mockingbird condition): The forest contains a mockingbird \(M\). 

One rumor has it that every bird of the forest is fond of at least one bird.
Another rumor has it that there is at least one bird that is not fond of any
bird. The interesting thing is that it is possible to settle the matter
completely by virtue of the given conditions \(C_{1}\) and \(C_{2}\).

Which of the two rumors is correct? 

\begin{center}
\rule{15cm}{0.5pt}
\end{center}
\setlength{\leftskip}{0pt}
\setlength{\rightskip}{0pt}

Do note that in this case, \(C\) and \(B\) do not refer to the \(C\) and \(B\)
combinators we've looked at previously.

The answer will be shown on the next page.

\newpage

Because of \(C_{1}\) and \(C_{2}\), we know that for every bird \(A\), there's
a bird---we'll call it \(C\)---that composes \(A\) with \(M\). We can say the
following:
\[Cx=A(Mx)=A(xx)\] 
If we now fill in \(C\) in place of \(x\), we get:
\[A(CC)=CC\]
We thus know that for any bird \(A\), \(A\) is fond of the bird \(CC\), where
\(C\) composes \(A\) with \(M\). Therefore rumour one is true and rumour two is
false.

The answer \textcite{smullyan2000} gives is a bit more verbose, but it comes
down to the same thing.

\section{Using lambda calculus for computation}

Now that we've gone throught lambda calculus and combinatory logic, it is
finally time to get to the good parts. I hope the way here wasn't too boring or
difficult. We will now look at how to use lambda calculus for computation.

\subsection{Boolean logic}

Let's look at a simple form of computation before jumping into numbers and
stuff. Let's start with binary logic. In it's simplest form, binary logic is
really just control flow: \emph{if A then B else C}. We want to have some
condition that chooses between two expressions. We know how to do that already
(see section \ref{combinatorylogic}). \(K\) choses the first of two expressions
and \(KI\) the latter. We can define true to be \(K\) and false to be \(KI\):
\[T:=K=\lambda ab.a\]
\[F:=KI=\lambda ab.b\]

We can now have some condition \(C\) that can either be \(T\) or \(F\) and
return an expression \(A\) or \(B\) depending on it.

But how can we do logic gates? Let's look at negation. We have already looked
at the flip combinator---\(C\) (section \ref{flipcombinator}), which does
exactly what we want. Thus, we can say:
\[NOT:=C=\lambda fab.fba\]

We could also do it another way. We already know how to do control flow, so we
could define a function that chooses the opposite of what the input is. In
simple programming terms: \emph{if P then F else T}. Or in lambda calculus:
\[NOT:=\lambda p.pFT\]

It depends which one's preferable. The \(C\) combinator is a bit more elegant
maybe, but you only get a function that is extentionally equal to \(T\) or
\(F\), while the other definition literally returns \(T\) or \(F\).

How do we define \(AND\)? We could define it in simple programming terms, and
then translate it to lambda caluclus. In simple programming terms, \(AND\)
would look like this: \emph{if P then (if Q then T else F) else F}. In lambda
calculus we would get:
\[AND:=\lambda pq.p(qTF)F\]
But this is a very naïve way of defining \(AND\). If you look closely at the
\(qTF\) part, you notice that it actually just returns whatever \(q\) is
anyway. Thus, we can say:
\[AND:=\lambda pq.pqF\]
You can also replace the remaining \(F\) with \(q\) if you want, because the
\(F\) will only be returned if \(q\) is \(F\)---in other words: \(q\) in that
case is alway false. So we can say:
\[AND:=\lambda pq.pqp\]

We can define \(OR\) in a very similar way. If the first argument is true, we
just return it, else we return the second argument:
\[OR:=\lambda pq.ppq\]

Defining \(XOR\) is very easy too. If the first argument is true, then we want
to return what the second argument is not, else we want to return what the
second argument is. In lambda calculus:
\[XOR:=\lambda pq.p(NOT\:q)q\]
Or:
\[XOR:=\lambda pq.p(qFT)q\]

We can define boolean equality (\(BEQ\)) similarly:
\[BEQ:=\lambda pq.pq(qFT)\]
Which you can read as: "If \(p\) is true, then return what \(q\) is, else
return what \(q\) is not."

\subsection{Church encodings}

\subsection{Recursion}

\subsection{Data structures}

\section{Functional programming (lambda calculus applied)}

\subsection{Why do we even care?}

\subsection{Lambda functions}

\subsection{Laziness}

\subsection{Types}

\subsection{Monads}

\subsection{Haskell}

\subsection{Comparison to with other paradigms}

\subsubsection{Declarative vs imperative}

\subsubsection{Usefulness vs conceptual purity}

\subsubsection{Meta programming}

\section{Functional programming in other paradigms}

\subsection{Lambda functions}

\subsection{Iterators}

\section{Possibilities for the future}

\subsection{Conceptual \emph{and} useful}

\section*{Afterword}
\addcontentsline{toc}{section}{Afterword}

\newpage
\printbibliography[heading=bibintoc, title={References}]
\end{document}
