\documentclass[a4paper, 11pt]{article}

%\title{\textbf{Lambda Calculus}\\ \Large and its Impact on Computer Science}
\title{Lambda Calculus and its Impact on Computer Science}
\author{Joris Klaasse Bos\\ Zaanlands Lyceum}

\begin{document}

\maketitle

\section{Prologue}

The reason I chose this subject is that it combines two things I enjoy:
abstract maths and computer science. I have been programming for about five to
six years now. I mainly enjoy low-level programming, so naturally C is my most
used language and I am most familiar with a simple procedural paradigm. Such a
paradigm is, however, not always very easy to use when working with very large
and complex systems. I, as many, started out with Object-Oriented Programming,
but I did not like that very much. Therefore, I have been exploring alternative
paradigms, including Data-Oriented Programming and Functional Programming. I am
quite familiar with Data-Oriented Programming and the Rust programming language
by now, but Functional Programming isn't something I ever really have got in to
yet. I did find out about lambda calculus and combinatory logic, which
intrigued me, but I haven’t got into it beyond a basic level of understanding.
That is why I decided to research it for this paper. 

\section{Introduction}

% Note to self: add footnote for defintion of calculus, and that algebra is
% limited

Lambda calculus is, as its name suggests, a calculus. A calculus is a system of
manipulating symbols, which by themselves don't have any semantic meaning, in a
way that is somehow meaningful. We all know algebra. Algebra itself doesn't
have an innate meaning, but we can use it to represent and solve real world
problems. Algebra, however, is limited. Not every problem can be represented in
algebra. There are many branches of mathematics that use different systems.
One example would be formal logic, which is used for logical operations on
booleans. Another such example is lambda calculus.

\subsection{A short history}

% Something about church and curry and turing machines

\subsection{The syntax}

Lambda calculus is all about unary anonymous functions. Such a function has no
name to identify it, takes only one input, and returns a single expression that
is only dependent on the input, so it doesn't have any outside state. A simple
function definition in lambda calculus looks as follows:
\[\lambda a.a\]

The lambda signifies a function. Everything following it will be part of that
function's definition. The \(a\) before the \(.\) is the name of the argument.
There is only one, because, as I said before, all functions in lambda calculus
are unary. Everything following the \(.\) is part of the return expression. The
funcion above is the identity function in lambda calculus; it just returns the
input. This is the equivalent of multiplying by one, or defining a function
like \(f(x)=x\), or multiplying a vector with the identity matrix; it does
nothing.

But how do we use this function? Well, just like defining a function, it is
quite simple. If you want to apply this function to a symbol, you just put it
parentheses behind the symbol. Something like this:
\[x(\lambda a.a)\]
Which evaluates to:
\[x\]

I have now basically explained the entire lambda calculus, it is really that
simple. I have explained abstraction (the functions), application (applying a
function to a symbol), and grouping (the parentheses), which is basically all
we need. You can also give names to expressions. We could name our identity
function \(I\) as follows: \[I:=\lambda a.a\] But this isn't really part of the
core lambda calculus anymore, just some syntactic sugar. This way, instead
of constantly having to write \(\lambda a.a\), we can just write \(I\). We have
now covered identifiers too.

\subsection{Combinatory logic}

% Something about combinatory logic and how it translates to lambda calculus

\end{document}
